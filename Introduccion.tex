\section{Introducción}

Cuando hablamos de automatización, es fundamental considerar las necesidades que se pretenden satisfacer en cualquier industria. Los empresarios, especialmente en el ámbito de la agricultura, buscan constantemente maneras de agilizar y modernizar sus procesos de producción. Esto responde a la creciente demanda de eficiencia, precisión y ahorro de recursos. En este contexto, es inevitable hacer referencia a la Revolución Industrial, un período que cambió radicalmente la sociedad, la economía y la tecnología. Esta revolución dio origen a la creación de máquinas capaces de realizar tareas que antes dependían exclusivamente de la mano de obra humana, y sentó las bases para los avances tecnológicos que hoy conocemos.

La automatización en la agricultura no es una excepción. De hecho, en la actualidad se concentra en la creación de máquinas inteligentes diseñadas para optimizar una amplia gama de actividades, desde el cultivo y riego hasta la recolección y clasificación de productos. Entre estas actividades, la clasificación de alimentos ocupa un lugar clave. Este proceso es esencial para asegurar que los productos que llegan al consumidor final cumplan con los estándares de calidad que el mercado exige, especialmente en un mundo donde la demanda de alimentos frescos y de alta calidad es cada vez mayor.

La importancia de la clasificación de alimentos está profundamente arraigada en nuestra cultura. Desde temprana edad, aprendemos la relevancia de una alimentación balanceada y de calidad, algo que se nos inculca a través de guías como el "Plato del Buen Comer". A lo largo de la vida, el acto de seleccionar alimentos en mercados o tiendas se convierte en una práctica cotidiana que, aunque sencilla en apariencia, es crucial para garantizar que lo que consumimos sea saludable y nutritivo. Al elegir frutas y verduras, por ejemplo, no solo tomamos en cuenta su apariencia, sino también su frescura, madurez y estado general. Este proceso manual de selección, que en muchas familias se convierte en una tradición, es una forma intuitiva de asegurar la calidad de los alimentos, pero también es ineficiente y propenso a errores cuando se trata de grandes volúmenes en entornos industriales.

Un claro ejemplo de esta problemática se presenta en la producción de cítricos, específicamente en la clasificación de limones. La calidad de estos productos varía considerablemente dependiendo de la temporada y del tipo de limón. El limón verde, que tiene una temporada de septiembre a junio, se distingue por su acidez, ausencia de semillas y apariencia externa, características que lo diferencian de otros tipos de limones. Si bien es relativamente sencillo para un agricultor o seleccionador experimentado identificar estas diferencias visuales, este proceso manual es lento y está sujeto a errores humanos. La clasificación de grandes volúmenes de limones basándose únicamente en observaciones visuales no solo consume tiempo, sino que también puede resultar en productos clasificados incorrectamente, lo que afecta negativamente tanto la calidad del producto como la rentabilidad de los productores.

Es en este punto donde la automatización puede marcar una diferencia sustancial. La implementación de sistemas de clasificación automatizados, basados en parámetros como el tamaño, el color y otras características visuales, permite optimizar el proceso y reducir significativamente los tiempos de inspección. A través de la automatización, se logra una clasificación más precisa y consistente, lo que a su vez mejora la calidad de los productos que llegan al consumidor final. Los sistemas automatizados no solo seleccionan con mayor precisión los limones que cumplen con los estándares de calidad, sino que también eliminan aquellos que no son aptos para la venta, lo que minimiza las pérdidas y optimiza el uso de los recursos disponibles.

El uso de tecnologías avanzadas, como la visión artificial, es una de las soluciones más prometedoras en este campo. Esta tecnología permite a las máquinas "ver" y analizar las características visuales de los productos, clasificándolos con una precisión que supera a la del ojo humano. La visión artificial, combinada con algoritmos de aprendizaje automático, puede identificar de manera eficiente atributos como el color, el tamaño, la madurez y otros factores clave que determinan la calidad de un limón. Estas innovaciones no solo mejoran la precisión en la clasificación, sino que también aumentan la velocidad del proceso, lo que es fundamental en un entorno de producción agrícola a gran escala.

En este proyecto, el objetivo principal es desarrollar un sistema automatizado que utilice la visión artificial para clasificar limones según su calidad, basado en parámetros predefinidos de color y tamaño. Este sistema no solo promete aumentar la eficiencia del proceso de clasificación, sino que también busca reducir el margen de error humano y garantizar que solo los limones en perfecto estado lleguen al mercado. De esta manera, se pretende contribuir tanto a la sostenibilidad del sector agrícola como a la satisfacción del consumidor final, quien espera productos frescos y de alta calidad.

La implementación de este tipo de sistemas puede tener un impacto significativo en la industria agrícola. A medida que los productores adoptan estas tecnologías, no solo se optimizan los procesos internos, sino que también se mejora la competitividad en el mercado, al poder ofrecer productos de mejor calidad de manera más eficiente. La automatización, por tanto, no es simplemente una tendencia, sino una necesidad creciente para los productores que buscan mantenerse al día con las demandas del mercado y aprovechar al máximo los recursos disponibles.