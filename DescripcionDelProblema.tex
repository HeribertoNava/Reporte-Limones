\section{Descripción del Problema}

La clasificación de limones sigue siendo, en gran medida, un proceso manual que enfrenta numerosas limitaciones. A pesar de los avances en la automatización en otras áreas de la agricultura, la clasificación de frutas como el limón todavía depende en gran parte de trabajadores que realizan evaluaciones visuales, lo que da lugar a varios problemas relacionados con la inconsistencia, la ineficiencia y los altos costos. Esta actividad manual conlleva errores humanos frecuentes que impactan directamente en la calidad del producto y, por ende, en la satisfacción de los clientes.

Un problema evidente en el proceso de clasificación manual es la falta de precisión y uniformidad. Los trabajadores encargados de clasificar los limones se basan en su experiencia y percepción personal para evaluar características como el tamaño, el color y la madurez de la fruta. Sin embargo, esta subjetividad genera inconsistencias en los resultados. Un trabajador puede considerar que un limón cumple con los estándares de calidad, mientras que otro puede clasificar el mismo limón de manera diferente. La falta de criterios estandarizados afecta negativamente la calidad general del producto y puede llevar a una mezcla de frutas de distintos grados de madurez y calidad en el mismo lote, lo que genera descontento entre los consumidores.

Otro desafío significativo es la incapacidad del método manual para manejar grandes volúmenes de producto de manera eficiente. A medida que la demanda de limones aumenta tanto a nivel local como internacional, los productores enfrentan la presión de clasificar mayores cantidades en menos tiempo. Sin embargo, el proceso manual es lento y requiere un esfuerzo considerable. Esto no solo retrasa la distribución de los productos, sino que también limita la capacidad de los productores para cumplir con los plazos establecidos. En un entorno competitivo, donde la frescura y calidad del producto son factores determinantes para el éxito, los productores que no puedan acelerar sus procesos de clasificación corren el riesgo de quedarse atrás.

La falta de estandarización en los criterios de calidad es otro aspecto preocupante. Cada trabajador puede aplicar diferentes parámetros al clasificar los limones, lo que resulta en una clasificación irregular. Esta variabilidad no solo afecta la calidad del producto, sino que también genera problemas en la cadena de suministro, ya que los distribuidores y minoristas confían en productos consistentes que cumplan con los estándares establecidos. Cuando los limones de menor calidad son mal etiquetados como productos de alta calidad, no solo se compromete la satisfacción del cliente, sino que también se producen pérdidas económicas tanto para los productores como para los distribuidores.

Uno de los mayores riesgos de la clasificación manual es la propensión a cometer errores, especialmente en situaciones de alta presión o cuando los trabajadores están cansados. Según Sánchez et al. (2022), hasta un 15\% de los limones clasificados manualmente son rechazados por defectos no detectados. Esta cifra representa una pérdida significativa, tanto en términos de ingresos como de calidad. Los errores humanos no solo afectan la reputación de los productores, sino que también generan una cantidad considerable de desperdicio. En un contexto donde la sostenibilidad y la reducción de residuos son prioritarias, estas pérdidas son inaceptables.

Los costos operativos asociados a la clasificación manual son igualmente alarmantes. La dependencia de mano de obra calificada conlleva altos costos salariales y de capacitación. Según estimaciones de López y Pérez (2023), los gastos de clasificación pueden representar hasta un 20\% del costo total de producción. Este margen de costos afecta directamente la rentabilidad de los productores, quienes deben encontrar formas de reducir gastos para mantener su competitividad en el mercado. En este sentido, la clasificación manual no es un método sostenible ni eficiente a largo plazo.

La creciente preocupación por la sostenibilidad también debe tenerse en cuenta. Ramírez y Torres (2020) señalan que hasta un 30\% de las frutas y verduras se desperdician debido a ineficiencias en la clasificación y el empaquetado. La incapacidad de detectar limones en mal estado antes de su distribución no solo resulta en pérdidas económicas, sino que también contribuye al aumento de los residuos en la cadena de suministro agrícola. En un contexto donde se prioriza la sostenibilidad ambiental, es crucial reducir el desperdicio y optimizar el uso de los recursos.

Asi mismo, la falta de adaptabilidad de las técnicas tradicionales frente a la creciente demanda de limones de alta calidad es otro obstáculo importante. La capacidad de los productores para competir en el mercado global se ve limitada por la ineficiencia de los métodos manuales. Un análisis de Gómez et al. (2021) sugiere que la implementación de sistemas automatizados es un paso necesario para satisfacer las expectativas de los consumidores, quienes demandan productos no solo frescos, sino también homogéneos en términos de calidad.

Frente a estos retos, la automatización, específicamente mediante el uso de visión artificial, se presenta como una solución viable. Un sistema automatizado de clasificación basado en esta tecnología tiene el potencial de superar las limitaciones del proceso manual, al aumentar la precisión y consistencia en la clasificación de los limones. La visión artificial permite que las máquinas analicen parámetros visuales como el color, tamaño y textura de los limones, garantizando una evaluación estandarizada y uniforme de la calidad del producto. Este sistema no solo reduciría el margen de error, sino que también aceleraría el proceso de clasificación, permitiendo a los productores manejar mayores volúmenes de producto en menos tiempo.

Además, la automatización reduce los costos operativos asociados a la mano de obra, lo que incrementa la rentabilidad de los productores. Al eliminar la necesidad de depender exclusivamente de trabajadores calificados, los sistemas automatizados permiten un mayor control sobre el proceso y una disminución significativa en los costos relacionados con salarios y capacitación. En este sentido, la automatización no solo mejora la eficiencia del proceso de clasificación, sino que también representa una inversión rentable para los productores a largo plazo.