\section{Antecedentes Teóricos}

La clasificación de productos agrícolas, como los cítricos, es crucial para garantizar la calidad del producto y su valor en el mercado. Con el aumento de la demanda de productos frescos y de alta calidad, la automatización de este proceso se ha convertido en una necesidad apremiante. Las tecnologías de visión artificial han emergido como herramientas clave en este ámbito, con investigaciones recientes que ofrecen perspectivas valiosas, pero que también presentan limitaciones.

\subsection{Importancia de la Clasificación en la Industria Agrícola}

La clasificación efectiva de los limones asegura su calidad y determina su valor comercial. Según Vargas et al. (2020), la adopción de sistemas automatizados puede mejorar la calidad del producto al reducir la variabilidad y los errores de clasificación asociados con los métodos manuales. Sin embargo, estas soluciones automatizadas aún pueden ser ineficaces si no utilizan técnicas avanzadas para la identificación de características específicas de los limones.

\subsection{Desarrollo de Tecnologías de Visión Artificial}

El avance en las tecnologías de visión artificial ha permitido la creación de sistemas de clasificación más eficientes. Guevara et al. (2020) destacan que los sistemas que utilizan algoritmos de aprendizaje profundo pueden procesar imágenes en tiempo real, mejorando la identificación de los limones según sus características visuales. No obstante, muchos de estos sistemas se centran en parámetros limitados, lo que puede afectar la precisión de la clasificación. Por ejemplo, algunos métodos no consideran adecuadamente las variaciones de color que indican el estado de madurez del fruto, lo que puede llevar a clasificaciones incorrectas y pérdidas de productos.

\subsection{Impacto en la Productividad y Sostenibilidad}

Las investigaciones también indican que la inversión en tecnologías automatizadas, aunque beneficiosa a largo plazo, no siempre es accesible para todos los productores. Vargas et al. (2020) señalan que, si bien la automatización mejora la productividad, muchos sistemas requieren una intervención constante de los operarios, lo que limita su eficacia. En contraste, el sistema propuesto en este estudio aborda directamente estas limitaciones al desarrollar una solución automatizada de clasificación de limones basada en visión artificial y colorimetría, utilizando un ESP32-CAM y un sensor de color.

\subsection{Solución Propuesta: Sistema Automatizado de Clasificación}

La propuesta consiste en un sistema automatizado de clasificación de limones que integra tecnologías avanzadas para evaluar la calidad de la fruta en función del color de la piel externa. A diferencia de los sistemas existentes que pueden ser limitados en su enfoque, este sistema se propone:

- Reducción significativa de ineficiencia: Se anticipa que el nuevo sistema reducirá el tiempo y el esfuerzo manual actualmente requeridos en la clasificación, minimizando la dependencia de operarios y permitiendo una clasificación más rápida y precisa.
  
- Mejor identificación de limones en mal estado: Gracias al uso de parámetros de color establecidos y a un algoritmo de procesamiento de imágenes que emplea técnicas de análisis de colorimetría y machine learning, se espera que el sistema logre identificar de manera precisa los limones en mal estado antes de su empaque. Esto no solo disminuirá la pérdida de productos, sino que también mejorará la calidad del producto final.

- Aumento en la satisfacción del consumidor: Al asegurar que únicamente los limones en su punto óptimo de madurez lleguen al mercado, se incrementará la satisfacción del consumidor final, lo que es fundamental para la competitividad en el sector agrícola.

\subsection{Objetivos del Sistema Automatizado}

Los objetivos específicos del sistema propuesto son:

1. Desarrollar un sistema automatizado que permita a los productores clasificar los limones según su calidad, reduciendo el error humano y optimizando el proceso de clasificación.
  
2. Diseñar algoritmos y módulos de procesamiento que clasifiquen los limones por color y calidad, mejorando la precisión en la evaluación visual.

3. Definir una metodología de evaluación de parámetros de color, que identifique con precisión la madurez y calidad de los limones mediante métricas estandarizadas.

4. Validar la fiabilidad del sistema mediante pruebas controladas y comparación de resultados con estándares de color existentes para limones.

5. Realizar pruebas de integración entre los componentes del sistema en un entorno de producción real, asegurando tiempos de respuesta rápidos y alta precisión en la clasificación de grandes volúmenes de limones.

Con este enfoque, se busca superar las limitaciones de las investigaciones anteriores, ofreciendo una solución más eficiente y adaptable a las necesidades del sector agrícola.
