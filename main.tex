% Actualizacion 2024: Logotipos de la UPV


\documentclass[12pt]{article}
\usepackage[left=2.5cm,top=2.0cm,right=2.5cm,bottom=3.0cm]{geometry}
\usepackage[utf8]{inputenc}
\usepackage[spanish]{babel}
%\usepackage[english, american]{babel}
%\usepackage[spanish,es-tabla]{babel}
\usepackage[linguistics]{forest}
\usepackage{amssymb, amsmath, amsbsy} % simbolitos
\usepackage{longtable} % para tablas largas
\usepackage{graphicx}
\usepackage{fancyhdr}
\usepackage{xcolor}
\usepackage{multirow}
\usepackage{listings}
\usepackage{caption}
\usepackage{subcaption}
%\usepackage{parskip}
\usepackage[skip=12pt plus1pt]{parskip}
\usepackage{pdfpages} % Incluir PDF en documento en LATEX
\usepackage{verbatim} % comentarios
\usepackage{algpseudocode}
\usepackage{algorithm}
\usepackage{pdflscape}
\usepackage{multirow}
\usepackage{afterpage}
\usepackage{array,booktabs,ragged2e}
%\newcolumntype{R}[1]{>{\RaggedLeft\arraybackslash}p{#1}}
\newcolumntype{L}[1]{>{\raggedright\let\newline\\\arraybackslash\hspace{0pt}}m{#1}}
\newcolumntype{C}[1]{>{\centering\let\newline\\\arraybackslash\hspace{0pt}}m{#1}}
\newcolumntype{R}[1]{>{\raggedleft\let\newline\\\arraybackslash\hspace{0pt}}m{#1}}

\floatname{algorithm}{Algoritmo}
\renewcommand{\listalgorithmname}{Lista de algoritmos}
\renewcommand{\algorithmicrequire}{\textbf{Entrada:}}
\renewcommand{\algorithmicensure}{\textbf{Salida:}}

% Comentario con respecto a las referencias:
% Por defecto este documento utiliza el formato IEEEtr para formatear las referencias. Si algun asesor requiere formato APA, solo comente la siguiente linea y descomente la linea debajo. Tambien para que las citas funcionen de manera adecuada, el paquete label debe tener las opciones mostradas, o en su defecto, el año en el listado no lo muestra correctamente \usepackage[english, american]{babel}

% Para utilizar el formato de citas IEEE y comentar los dos parrafos siguientes
\usepackage[backend=bibtex,sorting=none]{biblatex}
% Para utilizar el formato APA, sugiero comentar la linea anterior y descomentar las dos proximas lineas
%\usepackage[backend=biber,style=apa]{biblatex}
%\DeclareLanguageMapping{english}{american-apa}

\makeatletter
\DefineBibliographyExtras{spanish}{%
  \setcounter{smartand}{1}%
  \let\lbx@finalnamedelim=\lbx@es@smartand
  \let\lbx@finallistdelim=\lbx@es@smartand
}
\renewbibmacro*{name:delim:apa:family-given}[1]{%
  \ifnumgreater{\value{listcount}}{\value{liststart}}
    {\ifboolexpr{
       test {\ifnumless{\value{listcount}}{\value{liststop}}}
       or
       test \ifmorenames
     }
       {\printdelim{multinamedelim}}
       {\lbx@finalnamedelim{#1}}}
    {}}
\makeatother


% Estas lineas permiten romper los hipervinculos muy largos en las referencias!!!!
\setcounter{biburllcpenalty}{7000}
\setcounter{biburlucpenalty}{8000}
\addbibresource{x.bib} % ARCHIVO DE BIBLIOGRAFÍA


%\usepackage{url}
\usepackage[bookmarks=true,breaklinks=true,bookmarksopen=false,colorlinks=true,linkcolor=blue]{hyperref}
\usepackage[hyphenbreaks]{breakurl}
% Regla que define explicitamente que caracteres rompen los hipervinculos para separar las lineas
%https://es.overleaf.com/11089898rhgykrqyqytx
% Actualiza en automático la fecha de las citas de internet a la fecha de la compilación del documento
\usepackage{datetime}
\newdateformat{specialdate}{\twodigit{\THEDAY}-\twodigit{\THEMONTH}-\THEYEAR}
%\newdateformat{specialdate}{\twodigit{\THEDAY}-\THEYEAR}
\date{\specialdate\today}

\newcommand{\HRule}{\rule{\linewidth}{0.25mm}}


% CONSTANTES NECESARIAS PARA EL DOCUMENTO ---> MODIFIQUEN A SU CRITERIO
\newcommand{\ncarrera}            {Ingeniería en Tecnologías de la Información}
\newcommand{\nasesorinstitucional}{Dr. Said Polanco Martagón}



% ALUMNOS
\newcommand{\NombreAlumno}{Erika Daniela Mallozzi Martínez y Heriberto Geovanny Nava Lopez}
%Hombres cambien LA por EL
\newcommand{\elolaNombreAlumno}{la}  
\newcommand{\OA}           {a}  %Hombres cambien A por O
\newcommand{\Matricula}{2130128 y 2130213}





\newcommand{\NombreProyecto}{Prototipo de automatización de Clasificación de Limones por Calidad Utilizando
Visión Artificial Basada en el Color.}
\newcommand{\fechacarta}{26 de Abril de 2021}
\newcommand{\ncuatrimestre}{Mayo-agosto 2021}
\newcommand{\nevalador}{Dr. Said Polanco Martagón}
\newcommand{\FechaExposicion}{11 de Agosto de 2021}
\newcommand{\HoraExposionFormatoVenticuatroHoras}{10:00}
\newcommand{\elolaNombreEmpresa}{la}  %Si la empresa es femenino (por ejemplo universidad, usen la) o masculino (el instituto) pongan el
\newcommand{\organismoreceptor}   {Universidad Politécnica de Victoria}

% NOTA: Dos diagonales juntas (\\) indican un saldo de linea. En este caso particular hay 2 (el titulo se ajusta a tres lineas, porque es muy largo. Hacer las adecuaciones pertinentes
\newcommand{\NombreProyectoheader}     {Prototipo de automatización de Clasificación de Limones por Calidad Utilizando
Visión Artificial Basada en el Color}
\newcommand{\nasesorempresaria}   {Dr. Said Polanco Martagón}
\newcommand{\fechaPortada}               {Octubre de 2024}





\newcommand{\separacionCorta}{0.0cm}
\newcommand{\separacionLarga}{0.5cm}

\usepackage[overload]{textcase}
\newcommand{\iemph}[1]{\MakeTextUppercase{#1}}

\pagestyle{fancy}
\headheight 45pt
\fancyhead{} % Clear all header fields
\fancyhead[L]{\includegraphics[height=1.00cm]{UTYP.png}}%
\fancyhead[C]{\begin{center}\NombreProyectoheader\end{center}}%
%\fancyhead[R]{\includegraphics[height=1.5cm]{LogoUPV_2019.png}}%
\fancyhead[R]{\includegraphics[height=1.25cm]{LogoUPV_2023.png}}%
\fancyfoot[R]{\thepage} % Clear all footer fields 
\fancyfoot[C]{}
\fancyfoot[L]{}

\DefineBibliographyStrings{english}{%
  references = {Referencias},% replace "references" with "bibliography"  for `book`/`report`
}

\addto\captionsenglish{%
  \renewcommand{\figurename}{Figura}%
  \renewcommand{\tablename}{Tabla}%
} 

\usepackage{wallpaper}
 
 
%\renewcommand{\figurename}{Figura}
%\renewcommand{\tablename}{Tabla}

 
\begin{document}

%-----------------------------------------------------------------------------------------------------------------
% PAGINA 1 - PORTADA
\setcounter{page}{1}
\pagenumbering{roman}
\thispagestyle{empty}

\begin{center}

\begin{tabular}{cp{5cm}c}
\includegraphics[height=2.25cm]{UTYP.png} & 
& \includegraphics[height=2.25cm]{LogoUPV_2023.png}   \\
\end{tabular}

\Large \textbf{UNIVERSIDAD POLITÉCNICA DE VICTORIA}
\vspace{0.5cm}
\hrule
\vspace{0.1cm} 
\hrule
\vspace{0.5cm}


%\HRule \\[\separacionCorta]
\textbf{\iemph{\NombreProyecto}} \\[\separacionLarga]
%\Large \textbf{TESINA}
%\HRule \\[\separacionLarga]
T E S I N A \\

\textbf{\iemph{\ncarrera}} \\[\separacionLarga]

PRESENTA: \\[\separacionCorta]
%\textbf{\Capitalize{\NombreAlumno}\\[\separacionLarga]
\textbf{\iemph{\NombreAlumno}}\\[\separacionLarga]
%EN CUMPLIMIENTO DE \\[\separacionCorta]
%LA ESTADÍA DE LA CARRERA DE \\[\separacionCorta]


DIRECTOR \\[\separacionCorta]
\textbf{\iemph{\nasesorinstitucional}} \\[\separacionCorta]

CO-DIRECTOR \\[\separacionCorta]
\textbf{\iemph{\nasesorempresaria}} \\[\separacionCorta]

ORGANISMO RECEPTOR \\[\separacionCorta]
\textbf{\iemph{\organismoreceptor}} \\[\separacionLarga]

\end{center}
\begin{flushright}
\iemph{Ciudad Victoria, Tamaulipas, \fechaPortada}
\end{flushright}

\HRule 




%-----------------------------------------------------------------------------------------------------------------
% PAGINA 5 - RESUMEN EN ESPAÑOL

\clearpage
\section*{\centering Resumen}
\addcontentsline{toc}{section}{Resumen}
%\input{Resumen.tex}
\textbf{Palabras clave:} Tecnología, Sistema web, Desarrollar, Implementar, Empresas, Modulo.

%-----------------------------------------------------------------------------------------------------------------
% PAGINA 6 - RESUMEN EN INGLES

\clearpage
\section*{\centering Summary}
\addcontentsline{toc}{section}{Summary}
%\input{Summary.tex}
\textbf{Keywords}: Technology, Web system, Develop, Implement, Companies, Module.

%-----------------------------------------------------------------------------------------------------------------
% PAGINA 7 - INDICE

\clearpage
\addcontentsline{toc}{section}{Índice}
\renewcommand\contentsname{Índice}
\tableofcontents

%-----------------------------------------------------------------------------------------------------------------
% CAPITULOS


\clearpage
\pagenumbering{arabic}
\setcounter{page}{1}
%\input{Capitulo1.tex}

\clearpage
%\input{Capitulo2.tex}

\clearpage
\input{Capitulo3.tex}

\clearpage
\input{Capitulo4.tex}

\clearpage
\section{Descripción del Problema}

La clasificación de limones sigue siendo, en gran medida, un proceso manual que enfrenta numerosas limitaciones. A pesar de los avances en la automatización en otras áreas de la agricultura, la clasificación de frutas como el limón todavía depende en gran parte de trabajadores que realizan evaluaciones visuales, lo que da lugar a varios problemas relacionados con la inconsistencia, la ineficiencia y los altos costos. Esta actividad manual conlleva errores humanos frecuentes que impactan directamente en la calidad del producto y, por ende, en la satisfacción de los clientes.

Un problema evidente en el proceso de clasificación manual es la falta de precisión y uniformidad. Los trabajadores encargados de clasificar los limones se basan en su experiencia y percepción personal para evaluar características como el tamaño, el color y la madurez de la fruta. Sin embargo, esta subjetividad genera inconsistencias en los resultados. Un trabajador puede considerar que un limón cumple con los estándares de calidad, mientras que otro puede clasificar el mismo limón de manera diferente. La falta de criterios estandarizados afecta negativamente la calidad general del producto y puede llevar a una mezcla de frutas de distintos grados de madurez y calidad en el mismo lote, lo que genera descontento entre los consumidores.

Otro desafío significativo es la incapacidad del método manual para manejar grandes volúmenes de producto de manera eficiente. A medida que la demanda de limones aumenta tanto a nivel local como internacional, los productores enfrentan la presión de clasificar mayores cantidades en menos tiempo. Sin embargo, el proceso manual es lento y requiere un esfuerzo considerable. Esto no solo retrasa la distribución de los productos, sino que también limita la capacidad de los productores para cumplir con los plazos establecidos. En un entorno competitivo, donde la frescura y calidad del producto son factores determinantes para el éxito, los productores que no puedan acelerar sus procesos de clasificación corren el riesgo de quedarse atrás.

La falta de estandarización en los criterios de calidad es otro aspecto preocupante. Cada trabajador puede aplicar diferentes parámetros al clasificar los limones, lo que resulta en una clasificación irregular. Esta variabilidad no solo afecta la calidad del producto, sino que también genera problemas en la cadena de suministro, ya que los distribuidores y minoristas confían en productos consistentes que cumplan con los estándares establecidos. Cuando los limones de menor calidad son mal etiquetados como productos de alta calidad, no solo se compromete la satisfacción del cliente, sino que también se producen pérdidas económicas tanto para los productores como para los distribuidores.

Uno de los mayores riesgos de la clasificación manual es la propensión a cometer errores, especialmente en situaciones de alta presión o cuando los trabajadores están cansados. Según Sánchez et al. (2022), hasta un 15\% de los limones clasificados manualmente son rechazados por defectos no detectados. Esta cifra representa una pérdida significativa, tanto en términos de ingresos como de calidad. Los errores humanos no solo afectan la reputación de los productores, sino que también generan una cantidad considerable de desperdicio. En un contexto donde la sostenibilidad y la reducción de residuos son prioritarias, estas pérdidas son inaceptables.

Los costos operativos asociados a la clasificación manual son igualmente alarmantes. La dependencia de mano de obra calificada conlleva altos costos salariales y de capacitación. Según estimaciones de López y Pérez (2023), los gastos de clasificación pueden representar hasta un 20\% del costo total de producción. Este margen de costos afecta directamente la rentabilidad de los productores, quienes deben encontrar formas de reducir gastos para mantener su competitividad en el mercado. En este sentido, la clasificación manual no es un método sostenible ni eficiente a largo plazo.

La creciente preocupación por la sostenibilidad también debe tenerse en cuenta. Ramírez y Torres (2020) señalan que hasta un 30\% de las frutas y verduras se desperdician debido a ineficiencias en la clasificación y el empaquetado. La incapacidad de detectar limones en mal estado antes de su distribución no solo resulta en pérdidas económicas, sino que también contribuye al aumento de los residuos en la cadena de suministro agrícola. En un contexto donde se prioriza la sostenibilidad ambiental, es crucial reducir el desperdicio y optimizar el uso de los recursos.

Asi mismo, la falta de adaptabilidad de las técnicas tradicionales frente a la creciente demanda de limones de alta calidad es otro obstáculo importante. La capacidad de los productores para competir en el mercado global se ve limitada por la ineficiencia de los métodos manuales. Un análisis de Gómez et al. (2021) sugiere que la implementación de sistemas automatizados es un paso necesario para satisfacer las expectativas de los consumidores, quienes demandan productos no solo frescos, sino también homogéneos en términos de calidad.

Frente a estos retos, la automatización, específicamente mediante el uso de visión artificial, se presenta como una solución viable. Un sistema automatizado de clasificación basado en esta tecnología tiene el potencial de superar las limitaciones del proceso manual, al aumentar la precisión y consistencia en la clasificación de los limones. La visión artificial permite que las máquinas analicen parámetros visuales como el color, tamaño y textura de los limones, garantizando una evaluación estandarizada y uniforme de la calidad del producto. Este sistema no solo reduciría el margen de error, sino que también aceleraría el proceso de clasificación, permitiendo a los productores manejar mayores volúmenes de producto en menos tiempo.

Además, la automatización reduce los costos operativos asociados a la mano de obra, lo que incrementa la rentabilidad de los productores. Al eliminar la necesidad de depender exclusivamente de trabajadores calificados, los sistemas automatizados permiten un mayor control sobre el proceso y una disminución significativa en los costos relacionados con salarios y capacitación. En este sentido, la automatización no solo mejora la eficiencia del proceso de clasificación, sino que también representa una inversión rentable para los productores a largo plazo.

\clearpage
\section{Motivación}



En el sector agrícola, la optimización de los procesos de producción es un factor clave para mejorar la eficiencia y garantizar la calidad de los productos que llegan al mercado. La clasificación de frutas, como los limones, sigue siendo un proceso que en muchos casos se realiza de forma manual, lo que no solo genera altos costos laborales, sino que también introduce variabilidad y errores humanos en la evaluación de la madurez y calidad del producto. Estas ineficiencias pueden impactar de manera negativa en la cadena de suministro, afectando tanto a los productores como a los consumidores.

El presente proyecto surge de la necesidad de automatizar este proceso utilizando tecnologías accesibles y económicas, como la ESP32-CAM, para desarrollar un sistema de clasificación basado en la visión artificial. La motivación principal es aportar una solución que no solo disminuya los tiempos y costos asociados a la selección manual de limones, sino que también mejore la precisión y consistencia de los resultados. Esto permitirá a los productores mantener estándares de calidad más altos y constantes, minimizando pérdidas y maximizando el aprovechamiento de los recursos.

Además, la implementación de un sistema de clasificación automatizado tiene un impacto directo en la reducción del desperdicio de alimentos, ya que permite clasificar los productos de manera más precisa según su estado de madurez, lo que a su vez contribuye a una mayor sostenibilidad en la producción agrícola. La utilización de la ESP32-CAM, por su accesibilidad y capacidad para realizar tareas complejas de procesamiento de imágenes, también abre la puerta para que pequeños y medianos productores puedan adoptar soluciones tecnológicas avanzadas sin necesidad de realizar grandes inversiones en infraestructura.

El prototipo pretende no solo resolver un problema específico dentro de la industria agrícola, sino también fomentar el uso de tecnologías accesibles y aplicables en diferentes áreas de la agricultura, alineándose con las tendencias actuales de automatización e innovación tecnológica.


\clearpage
\section{Justificación}

El modelo de innovación considerado para el proyecto \textit{"Automatización de Clasificación de Limones por Calidad Utilizando Visión Artificial Basada en el Color"} es la innovación incremental. 

La innovación incremental se refiere a las mejoras y ajustes realizados a productos, servicios o procesos existentes, en lugar de crear algo completamente nuevo. Esta forma de innovación busca optimizar lo que ya se tiene, aumentando la eficiencia y la efectividad de los métodos actuales.

\begin{itemize}
    \item \textbf{Mejoras en Procesos Existentes:} El proyecto se basa en la automatización de un proceso de clasificación que ya existe, el cual es tradicionalmente manual. La implementación de visión artificial y tecnología de sensores representa una mejora progresiva que busca optimizar la clasificación de limones sin revolucionar completamente el sistema agrícola.
    
    \item \textbf{Aumento de Eficiencia:} A través de esta innovación, se espera mejorar la eficiencia del proceso de clasificación, reduciendo tiempos de operación y costos, al mismo tiempo que se mantiene la esencia del proceso agrícola. La adopción de nuevas tecnologías que complementan y mejoran las prácticas existentes es un rasgo característico de la innovación incremental.
    
    \item \textbf{Reducción de Riesgos:} Al introducir mejoras graduales en lugar de cambios radicales, se minimizan los riesgos asociados con la implementación de tecnologías completamente nuevas. Esto permite a los productores adaptar sus operaciones de manera más controlada y gestionar mejor los impactos en su cadena de suministro.
    
    \item \textbf{Satisfacción del Cliente:} Las mejoras en la calidad del producto resultantes de una clasificación más precisa y eficiente pueden aumentar la satisfacción del consumidor. Esto es un objetivo clave de la innovación incremental, que busca realizar ajustes que beneficien a los usuarios finales sin alterar drásticamente la oferta existente.
    
    \item \textbf{Facilitación de la Adopción:} La innovación incremental tiende a ser más fácilmente aceptada por los usuarios, ya que se construye sobre lo que ya conocen. Esto permite una transición más suave hacia la automatización, ya que los trabajadores agrícolas y otros involucrados en la cadena de suministro pueden aprender y adaptarse a los nuevos procesos sin experimentar un cambio drástico en su entorno de trabajo.
\end{itemize}


\clearpage
\section{Alcances y limitaciones}

Para este proyecto se utilizarán diversas herramientas, tecnologías, hardware y software que permitirán una integración efectiva del sistema.

\subsection{Hardware}
\begin{itemize}
    \item \textbf{ESP32-CAM:} Este módulo se utilizará para capturar imágenes de los limones en tiempo real. Su integración se realizará mediante conexión Wi-Fi, permitiendo la transmisión de datos y la comunicación con un servidor o aplicación de procesamiento de imágenes.
    
    \item \textbf{Sensor de Color TCS3200:} Este sensor permitirá medir los parámetros de color de la piel de los limones. Se integrará directamente con el ESP32-CAM, enviando datos sobre el color detectado para su análisis.
    
    \item \textbf{Plataforma de Prueba (PCB):} Se utilizará una placa de circuito impreso personalizada para montar el ESP32-CAM y el sensor de color. Esto garantizará una conexión estable y facilitará el manejo del hardware en entornos de producción.
\end{itemize}

\subsection{Software}
\begin{itemize}
    \item \textbf{Python:} Se utilizará como lenguaje de programación principal para el procesamiento de imágenes y la implementación de algoritmos de clasificación. Python es ideal por su amplia variedad de bibliotecas y su compatibilidad con tecnologías de visión artificial.
    
    \item \textbf{OpenCV:} Esta biblioteca de visión por computadora será fundamental para procesar las imágenes capturadas por el ESP32-CAM. Se encargará de realizar la segmentación y análisis de color, así como de aplicar algoritmos de clasificación.
    
    \item \textbf{Flask:} Este framework de microservicio se utilizará para crear una API que permita la comunicación entre el hardware y el software. La API recibirá datos del sensor y de la cámara, procesará la información y devolverá los resultados de clasificación.
\end{itemize}

\subsection{Tecnologías y APIs}
\begin{itemize}
    \item \textbf{RESTful API:} Se implementará una API RESTful para permitir la comunicación entre el módulo ESP32-CAM, el servidor y cualquier aplicación frontend que muestre los resultados. Esta API será responsable de recibir imágenes y datos del sensor, procesarlos y devolver la clasificación de calidad.
    
    \item \textbf{Firebase:} Se utilizará Firebase como base de datos en tiempo real para almacenar los resultados de la clasificación y permitir el acceso a los datos desde diferentes dispositivos. Firebase facilitará la sincronización de datos entre el servidor y las aplicaciones que utilicen la información.
\end{itemize}

\subsection{Integración}
\begin{itemize}
    \item \textbf{Comunicación en Tiempo Real:} El ESP32-CAM enviará imágenes y datos del sensor de color a la API RESTful implementada en Flask. Esta API procesará la información utilizando OpenCV y devolverá la clasificación de calidad de los limones.
    
    \item \textbf{Almacenamiento y Acceso a Datos:} Los resultados de clasificación se almacenarán en Firebase, permitiendo su acceso desde aplicaciones móviles o web. Esto facilitará la consulta de datos históricos y el monitoreo de la calidad de los limones clasificados.
    
    \item \textbf{Interfaz de Usuario:} Se desarrollará una interfaz web sencilla que se conectará a la API para mostrar los resultados de clasificación en tiempo real. Los usuarios podrán visualizar los datos almacenados en Firebase, facilitando la toma de decisiones en la cadena de suministro.
\end{itemize}


\clearpage
\input{Capitulo5.tex}

\clearpage
\input{Capitulo6.tex}

\clearpage
\addcontentsline{toc}{section}{Índice de figuras}
\renewcommand\listfigurename{Índice de figuras}

\listoffigures

\clearpage
\addcontentsline{toc}{section}{Índice de cuadros}
\renewcommand\listtablename{Índice de cuadros}
\listoftables

\clearpage
\addcontentsline{toc}{section}{Índice de algoritmos}
\renewcommand\listalgorithmname{Índice de algoritmos}
\listofalgorithms

%-----------------------------------------------------------------------------------------------------------------
% REFERENCIAS


\clearpage
%Let's cite! The Einstein's journal paper \cite{dirac} and the Dirac's 
%book \cite{einstein} are physics related items. 

%\Urlmuskip=0mu plus 1mu\relax
\addcontentsline{toc}{section}{Referencias} 
\printbibliography
 
\end{document}
