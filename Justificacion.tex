\section{Justificación}

La clasificación de frutas por calidad es un proceso esencial dentro de la industria agrícola, ya que garantiza que los productos lleguen al mercado con los estándares adecuados de madurez y apariencia. Sin embargo, la clasificación manual presenta diversas limitaciones, como la variabilidad en la percepción humana, el alto costo laboral y la baja eficiencia en términos de tiempo. Ante esta problemática, surge la necesidad de implementar soluciones tecnológicas que optimicen este proceso.

La justificación de este proyecto radica en la creciente demanda de métodos más eficientes, precisos y sostenibles dentro del sector agrícola. Un sistema automatizado de clasificación por color, basado en visión artificial, puede mejorar significativamente la consistencia y velocidad del proceso, eliminando el factor de error humano y reduciendo los costos asociados a la mano de obra. Al utilizar un dispositivo accesible y económico como la ESP32-CAM, este proyecto también se justifica por su capacidad para democratizar el acceso a tecnologías avanzadas, permitiendo que pequeños y medianos productores agrícolas puedan adoptar soluciones tecnológicas sin necesidad de grandes inversiones en infraestructura.

Además, la clasificación automática de limones por color puede contribuir a una reducción significativa del desperdicio de alimentos. Al detectar de manera precisa el estado de madurez de los frutos, es posible evitar que productos subóptimos sean desechados, lo que a su vez tiene un impacto positivo en la sostenibilidad de la cadena de suministro. Esta iniciativa se alinea con objetivos de desarrollo sostenible, como la producción responsable y la innovación en la industria, lo que refuerza la relevancia de este proyecto no solo a nivel local, sino también global.

Este proyecto se justifica por su potencial para optimizar un proceso clave dentro de la producción agrícola, ofreciendo una solución tecnológica eficiente, accesible y con impacto positivo tanto en la rentabilidad de los productores como en la sostenibilidad del sistema alimentario.
