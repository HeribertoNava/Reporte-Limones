\section{Justificación}

El modelo de innovación considerado para el proyecto \textit{"Automatización de Clasificación de Limones por Calidad Utilizando Visión Artificial Basada en el Color"} es la innovación incremental. 

La innovación incremental se refiere a las mejoras y ajustes realizados a productos, servicios o procesos existentes, en lugar de crear algo completamente nuevo. Esta forma de innovación busca optimizar lo que ya se tiene, aumentando la eficiencia y la efectividad de los métodos actuales.

\begin{itemize}
    \item \textbf{Mejoras en Procesos Existentes:} El proyecto se basa en la automatización de un proceso de clasificación que ya existe, el cual es tradicionalmente manual. La implementación de visión artificial y tecnología de sensores representa una mejora progresiva que busca optimizar la clasificación de limones sin revolucionar completamente el sistema agrícola.
    
    \item \textbf{Aumento de Eficiencia:} A través de esta innovación, se espera mejorar la eficiencia del proceso de clasificación, reduciendo tiempos de operación y costos, al mismo tiempo que se mantiene la esencia del proceso agrícola. La adopción de nuevas tecnologías que complementan y mejoran las prácticas existentes es un rasgo característico de la innovación incremental.
    
    \item \textbf{Reducción de Riesgos:} Al introducir mejoras graduales en lugar de cambios radicales, se minimizan los riesgos asociados con la implementación de tecnologías completamente nuevas. Esto permite a los productores adaptar sus operaciones de manera más controlada y gestionar mejor los impactos en su cadena de suministro.
    
    \item \textbf{Satisfacción del Cliente:} Las mejoras en la calidad del producto resultantes de una clasificación más precisa y eficiente pueden aumentar la satisfacción del consumidor. Esto es un objetivo clave de la innovación incremental, que busca realizar ajustes que beneficien a los usuarios finales sin alterar drásticamente la oferta existente.
    
    \item \textbf{Facilitación de la Adopción:} La innovación incremental tiende a ser más fácilmente aceptada por los usuarios, ya que se construye sobre lo que ya conocen. Esto permite una transición más suave hacia la automatización, ya que los trabajadores agrícolas y otros involucrados en la cadena de suministro pueden aprender y adaptarse a los nuevos procesos sin experimentar un cambio drástico en su entorno de trabajo.
\end{itemize}
