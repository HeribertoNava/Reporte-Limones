\section{Estado del Arte}

La clasificación automática de frutas ha sido un tema de investigación activo durante las últimas décadas debido a la creciente necesidad de optimizar los procesos de selección en la industria agrícola. En particular, el uso de visión artificial ha permitido importantes avances en la automatización de la clasificación de frutas, mejorando tanto la eficiencia como la precisión en la evaluación de calidad.

\subsection{Avances en visión artificial para la clasificación de frutas}

Durante los últimos cinco años, la visión artificial ha experimentado avances significativos, impulsados por el desarrollo de cámaras más asequibles y la mejora de algoritmos de procesamiento de imágenes. Investigaciones recientes han utilizado técnicas de procesamiento digital de imágenes, aprendizaje automático y redes neuronales convolucionales (CNN) para analizar parámetros externos de las frutas, como color, textura y forma. Estos estudios han demostrado la efectividad de la visión artificial para clasificar una amplia variedad de productos agrícolas, incluyendo manzanas, tomates, y cítricos, entre otros.

Por ejemplo, estudios como los de Xie et al. (2021) demostraron el uso de redes neuronales profundas para la clasificación de cítricos en función de su madurez, utilizando el color como uno de los principales criterios. Otros trabajos, como el de Zhang et al. (2020), han desarrollado modelos de clasificación multietiqueta que integran tanto el color como la textura para obtener una evaluación más precisa de la calidad del fruto.

En el caso específico de los limones, la clasificación por color ha sido objeto de interés debido a su importancia en la determinación del grado de madurez. Varias investigaciones han señalado que el color externo es uno de los indicadores más fiables de la madurez en cítricos, y los sistemas de visión artificial basados en este criterio han demostrado ser altamente eficientes para este propósito. No obstante, la mayoría de estos estudios utilizan cámaras de alta resolución y sistemas de procesamiento avanzados, lo que incrementa el costo y la complejidad del sistema.

\subsection{Uso de sistemas embebidos para la clasificación de frutas}

En los últimos años, los dispositivos de bajo costo como la ESP32-CAM han comenzado a atraer la atención de los investigadores debido a su capacidad para implementar sistemas de visión artificial en entornos con recursos limitados. A diferencia de los sistemas más costosos, la ESP32-CAM ofrece una solución accesible para pequeños productores agrícolas que no pueden invertir en infraestructura tecnológica avanzada. Sin embargo, las investigaciones sobre su aplicación en la clasificación de frutas aún son limitadas.

A pesar de su bajo costo, estudios recientes han demostrado que la ESP32-CAM, combinada con algoritmos de procesamiento optimizados, puede ofrecer resultados satisfactorios en tareas de clasificación basadas en color. Un ejemplo es el trabajo de García et al. (2022), donde se utilizó la ESP32-CAM para clasificar manzanas por color con una precisión considerable, lo que sugiere su potencial para la clasificación de otros frutos como los limones.

\subsection{Desafíos y oportunidades}

A pesar de los avances en la clasificación automática de frutas mediante visión artificial, aún existen varios desafíos. La variabilidad de las condiciones de iluminación, la presencia de defectos físicos no relacionados con el color, y las limitaciones en la capacidad de procesamiento de los sistemas embebidos son factores que pueden afectar la precisión de los modelos de clasificación. Sin embargo, estos desafíos también presentan oportunidades para la innovación, especialmente en el contexto de sistemas accesibles como el que se plantea en este proyecto.

El presente trabajo se sitúa en un contexto en el que se busca democratizar el uso de tecnologías avanzadas en la industria agrícola, permitiendo a pequeños productores acceder a soluciones tecnológicas eficientes y de bajo costo. Al utilizar la ESP32-CAM como base, este proyecto pretende aportar una solución accesible y eficiente que, aunque centrada en la clasificación por color, podría expandirse para incluir otros parámetros de calidad en futuras investigaciones.

Aunque existen desarrollos recientes significativos en la automatización de la clasificación de frutas mediante visión artificial, este proyecto se diferencia al enfocarse en un dispositivo accesible como la ESP32-CAM. Se espera que, al optimizar su uso para la clasificación de limones por color, se contribuya a llenar el vacío existente en la investigación aplicada a pequeños productores con recursos limitados.
